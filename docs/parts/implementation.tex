\section{Implementazione}

L'implementazione è stata realizzata in \textit{Python} cercando di utilizzare il meno possibile
librerie esterne. In particolare le librerie utilizzate sono:
\begin{itemize}
      \item \texttt{matplotlib} per la visualizzazione dei grafici delle analisi effettuate;
      \item \texttt{graphviz} per la visualizzazione delle reti bayesiane risultanti.
\end{itemize}

Nel modulo \texttt{structure\_learning} è presente il sotto-modulo \texttt{networks} usato per
gestire le reti bayesiane. In questo sotto-modulo sono presenti le classi:
\begin{enumerate}
      \item \texttt{BayesianNode} che rappresenta un nodo della rete bayesiana e quindi una variabile aleatoria,
            contenendo le informazioni riguardanti le dipendenze condizionali (e quindi sia i padri che i figli di quel nodo),
            e la profondità del nodo nella rete;
      \item \texttt{BayesianNetwork} che rappresenta una rete bayesiana insieme ai suoi nodi; da notare sono i metodi
            \texttt{get\_colliders} e \texttt{count\_diff\_colliders}: il primo restituisce i nodi collider della rete bayesiana; 
            il secondo, data un'altra rete bayesiana sotto forma di \texttt{BayesianNetwork}, restituisce il numero di collider 
            in eccesso e in difetto della rete data rispetto a quella su cui è stato chiamato il metodo; questo permette di valutare
            quanto due reti bayesiane siano simili tra loro, confrontando le loro indipendenze condizionali\footfullcite{VP:1990};
            inoltre il metodo \texttt{draw\_graph} permette di salvare un'immagine della rete bayesiana utilizzando la libreria
            \texttt{graphviz};
      \item \texttt{CPTBayesianNode} estensione della classe \texttt{BayesianNode} che aggiunge la tabella delle probabilità
            condizionali per quel nodo, quindi per ogni possibile combinazione di valori dei padri e della variabile stessa assegna un
            valore di probabilità;
      \item \texttt{CPTBayesianNetwork} estensione della classe \texttt{BayesianNetwork} che permette di gestire solo nodi di tipo
            \texttt{CPTBayesianNode} e aggiunge il metodo \texttt{generate\_random\_sample} che genera un singolo campione casuale
            della distribuzione rappresentata dalla rete bayesiana utilizzando la libreria \texttt{random}, in maniera tale che ogni
            campione sia indipendente e identicamente distribuito rispetto agli altri.
\end{enumerate}

Inoltre nello stesso sotto-modulo viene definita una funzione per generare una
\texttt{CPTBayesianNetwork} a partire da un file di testo in formato \texttt{.net}.

Proseguendo ad analizzare gli altri elementi del modulo \texttt{structure\_learning} si hanno:
\begin{enumerate}[resume]
      \item \texttt{Samples} classe contenitore di un insieme di campioni generati da una rete bayesiana, insieme alla rete
            che li ha generati;
      \item \texttt{Heuristic} classe per rappresentare la funzione euristica utilizzata per l'apprendimento della struttura,
            che d'ora in poi prenderà il nome di \textit{euristica fattoriale}
            indicata nell'articolo come $g(i, \pi_i)$\footfullcite{HEURISTIC:CH:1992}; l'implementazione proposta prevede il precalcolo
            di tutti i fattoriali fino a $m + r - 1$ dove $m$ è il numero di campioni generati e $r$ la massima dimensione tra i domini
            delle variabili aleatorie, e questi valori vengono salvati nelle istanze di tale classe; per convenienza tale classe viene anche
            usata per contenere un'istanza della classe \texttt{Samples} usata dall'algoritmo di apprendimento;
      \item \texttt{LogaritmicHeuristic} estensione della classe \texttt{Heuristic} che nell'articolo viene proposta come
            alternativa per ridurre i tempi di calcolo della funzione euristica: infatti calcola $\log(g(i, \pi_i))$ quindi utilizza
            somme e sottrazioni al posto di moltiplicazioni e divisioni\footfullcite{LOGHEURISTIC:CH:1992}; rimane comunque il precalcolo dei
            \textbf{logaritmi} dei fattoriali;
      \item la funzione \texttt{k2} che implementa l'algoritmo \textit{K2}\footnotemark[3] per l'apprendimento
            della struttura della rete bayesiana, prendendo come parametri un'ordinamento delle variabili
            aleatorie, l'euristica da utilizzare contenente anche i campioni generati dalla rete bayesiana, e
            il numero massimo di genitori per ogni nodo; restituisce sempre un'istanza della classe
            \texttt{BayesianNetwork} che rappresenta la rete bayesiana contenente le sole dipendenze
            condizionali.
\end{enumerate}
