\section{Conclusioni}

Dobbiamo ribadire che tutti i test svolti assumono che l'ordine delle variabili preso in
considerazione dall'algoritmo sia ottimale, ovvero che ogni nodo sia figlio solamente di nodi
precedenti rispetto all'ordinamento; questo è un requisito molto forte che non è sempre verificato,
infatti in scenari di apprendimento di reti bayesiane spesso abbiamo a disposizione solamente un
insieme di campioni i.i.d.. A tal proposito l'articolo propone (in maniera generica) di sfruttare
un qualsiasi algoritmo di ricerca locale per individuare tra diversi ordinamenti casuali di
variabili il migliore eseguendo \textit{K2} su di esso e determinando un punteggio per la struttura
generata\footfullcite{LSEARCH:CH:1992}.